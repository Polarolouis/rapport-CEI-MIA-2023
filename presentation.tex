\documentclass{beamer}
\usetheme{Boadilla}

% importations
\usepackage[french]{babel} % pour dire que le texte est en francais
\usepackage{csquotes}
\usepackage[T1]{fontenc} % pour les font postscript
\usepackage[cyr]{aeguill} % Police vectorielle TrueType, guillemets francais
\usepackage{epsfig} % pour gérer les images
\usepackage{amsmath,amsthm} % très bon mode mathématique
\usepackage{amsfonts,amssymb,bm, bbold}% permet la definition des ensembles
\usepackage{algorithm2e} % pour les algorithmes
\usepackage{algpseudocode} % pour les algorithmes
\usepackage{float} % pour le placement des figure
\usepackage{url} % pour une gestion efficace des url
\usepackage{hyperref}  % pour les hyperliens dans le document
\usepackage{tikz} % For graph plots
% bibliographie
\usepackage[style=apa]{biblatex}
\addbibresource{references.bib}


% Tikz
%% Tikz Related
\usetikzlibrary{calc,shapes,backgrounds,arrows,automata,shadows,positioning}
\usetikzlibrary{arrows,shapes,positioning,shadows,trees,calc,backgrounds,automata,positioning}



\tikzset{
    basic/.style  = {draw, text width=3cm, font=\sffamily, rectangle},
    root/.style   = {basic, rounded corners=2pt, thin, align=center,
            fill=green!30},
    level 2/.style = {basic, rounded corners=6pt, thin,align=center, fill=green!60,
            text width=8em},
    level 3/.style = {basic, thin, align=left, fill=pink!60, text width=3.5cm}
}
% Couleurs
% pour tickz multilevel
\definecolor{redorg}{RGB}{215,48,39}
\definecolor{orangeorg}{RGB}{253,174,97}

\definecolor{blueind}{RGB}{69,117,233}
\definecolor{cyanind}{RGB}{116,173,209}
\definecolor{electricblue}{RGB}{125, 249, 255}

\definecolor{greenind}{RGB}{112,130,56}

\definecolor{burntorange}{RGB}{204, 85, 0}
\definecolor{goldenyellow}{RGB}{255, 192, 0}
\definecolor{peach}{RGB}{255, 229, 180}

\definecolor{gray}{RGB}{128,128,128}


\title{Séminaire des stagiaires}
\subtitle{Adaptation de colSBM aux réseaux bipartites}
\author[L. Lacoste]{Louis \textsc{Lacoste}}
\date{29 juin 2023}

\begin{document}

% titre
\begin{frame}
\maketitle
\end{frame}

\section{Contexte du modèle}
\begin{frame}
    \frametitle{Contexte écologique}
    \begin{itemize}
        \item De nombreux réseaux disponibles \parencite{WebLifeEcological} et décrivant des interactions similaires
        \item Re-grouper les réseaux selon leur similarité (\emph{clustering} de réseaux)
        \item Déterminer des structures d'interactions fines de manière agnostique
        \item Vérifier si le regroupement est lié à des co-variables
    \end{itemize}
\end{frame}
\begin{frame}
    \frametitle{Réseaux bipartites\footnote{Ou \emph{bipartis}. Voir \cite{larousseDefinitionsBipartiBipartite}.}}
    \begin{columns}[c]
    \begin{column}{0.48\textwidth}
        \centering
        Réseau bipartite\\
    \begin{tikzpicture}[scale=.6]
        \tikzstyle{every edge}=[-,>=stealth',shorten >=1pt,auto,draw,line width=1.5pt]
        \tikzstyle{every state}=[draw, text=white,scale=0.95, transform shape]
        \tikzstyle{every state}=[draw=none,text=white,scale=0.75, transform shape]
        \tikzstyle{every node}=[fill=blueind]
    
        \node[state, draw=black!50] (A1) at (0,5) {\textbf{R1}};
        \node[state, draw=black!50] (A2) at (2.5,5) {\textbf{R2}};
        \node[state, draw=black!50] (A3) at (5,5) {\textbf{R3}};
    
        \tikzstyle{every node}=[fill=greenind, shape=rectangle]
        \tikzstyle{every state}=[draw=none,text=white,scale=0.75, transform shape, shape=rectangle]
        \node[state, draw=black!50] (B1) at (0,0) {\textbf{C1}};
        \node[state, draw=black!50] (B2) at (1.25,0) {\textbf{C2}};
        \node[state, draw=black!50] (B3) at (2.5,0) {\textbf{C3}};
        \node[state, draw=black!50] (B4) at (3.75,0) {\textbf{C4}};
        \node[state, draw=black!50] (B5) at (5,0) {\textbf{C5}};
        \path (A1) edge [] (B1);
        \path (A1) edge  (B2);
        \path (A1) edge  (B3);
        \path (A1) edge  (B4);
        \path (A2) edge  (B3);
        \path (A2) edge  (B4);
        \path (A3) edge  (B5);
        \path (A2) edge  (B5);
    \end{tikzpicture}
    \end{column}
    \hfill
    \begin{column}{0.48\linewidth}
    Matrice d'incidence
    \smallskip
    $B=\left(
    \begin{array}{rrrrr}
    1 &   1 &   1 &   1 &   0 \\ 
        0 &   0 &   1 &   1 &   1 \\ 
        0 &   0 &   0 &   0 &   1 \\  
    \end{array}\right)
    $\\
    \end{column}
    \end{columns}
    \smallskip
    Permet de décrire toute interaction impliquant deux agents dont les rôles
    sont de natures différentes.\\
    Par exemple : hôtes-parasites, plantes-pollinisateurs, graines-disperseurs \dots
\end{frame}
\begin{frame}
    \frametitle{Latent Block Model (LBM)}
    Proposé par \cite{govaertEMAlgorithmBlock2005}.

    \begin{figure}[H]
        \center
        \begin{tikzpicture}[scale=.6]
            \tikzstyle{every state}=[draw, text=white,scale=0.95, transform shape]
            \tikzstyle{every state}=[draw=none,text=white,scale=0.75, transform shape]
            
            \tikzstyle{every node}=[fill=blueind]
            \node[state, draw=white, fill=none, text=black, scale=2] (pi1) at (1,5.7) {\textbf{$\pi_{{\color{blueind}\bullet}}$}};
            \node[state, draw=black!50] (R11) at (0,5) {\textbf{R11}};
            \node[state, draw=black!50] (R12) at (1,5) {\textbf{R12}};
            \node[state, draw=black!50] (R13) at (2,5) {\textbf{R13}};
            
            \tikzstyle{every node}=[fill=cyanind]
            \node[state, draw=white, fill=none, text=black, scale=2] (pi2) at (6.5,5.7) {\textbf{$\pi_{{\color{cyanind}\bullet}}$}};
            \node[state, draw=black!50] (R21) at (6,5) {\textbf{R21}};
            \node[state, draw=black!50] (R22) at (7,5) {\textbf{R22}};
            
            \tikzstyle{every node}=[fill=electricblue]
            \node[state, draw=white, fill=none, text=black, scale=2] (pi3) at (10,5.7) {\textbf{$\pi_{{\color{electricblue}\bullet}}$}};
            \node[state, draw=black!50] (R31) at (10,5) {\textbf{R31}};
        
            \tikzstyle{every node}=[fill=burntorange, shape=rectangle]
            \node[state, draw=white, fill=none, text=black, scale=2] (pi3) at (0.5,-0.7) {\textbf{$\rho_{{\color{burntorange}\bullet}}$}};
            \tikzstyle{every state}=[draw=none,text=white,scale=0.75, transform shape, shape=rectangle]
            \node[state, draw=black!50] (B1) at (0,0) {\textbf{C11}};
            \node[state, draw=black!50] (B2) at (1,0) {\textbf{C12}};
            \tikzstyle{every node}=[fill=goldenyellow, shape=rectangle]
            \node[state, draw=white, fill=none, text=black, scale=2] (pi3) at (4.5,-0.7) {\textbf{$\rho_{{\color{goldenyellow}\bullet}}$}};
            \node[state, draw=black!50] (B3) at (4,0) {\textbf{C21}};
            \node[state, draw=black!50] (B4) at (5,0) {\textbf{C22}};
            \tikzstyle{every node}=[fill=peach, shape=rectangle]
            \node[state, draw=white, fill=none, text=black, scale=2] (pi3) at (10,-0.7) {\textbf{$\rho_{{\color{peach}\bullet}}$}};
            \node[state, draw=black!50] (B5) at (10,0) {\textbf{C31}};
    
            \tikzstyle{every edge}=[-,>=stealth',shorten >=1pt,auto,draw,line width=1.5pt,draw opacity=0.2]
    
            \path (R11) edge[-,>=stealth',shorten >=1pt,auto,draw=gray,line width=1.5pt, fill=gray, opacity=1] node[anchor=center, fill=none] {$\alpha_{{\color{blueind}\bullet}{\color{burntorange}\bullet}}$} (B1);
            \path (R11) edge (B2);
            \path (R11) edge  (B3);
            \path (R11) edge  (B4);
    
            \path (R12) edge [] (B1);
            \path (R12) edge  (B2);
            \path (R12) edge  (B3);
            \path (R12) edge  (B4);
    
            \path (R13) edge [] (B1);
            \path (R13) edge  (B2);
            \path (R13) edge  (B3);
            \path (R13) edge[-,>=stealth',shorten >=1pt,auto,draw=gray,line width=1.5pt, fill=gray, opacity=1] node[anchor=center, fill=none] {$\alpha_{{\color{blueind}\bullet}{\color{goldenyellow}\bullet}}$} (B4);
    
            \path (R21) edge[-,>=stealth',shorten >=1pt,auto,draw=gray,line width=1.5pt, fill=gray, opacity=1] node[anchor=center, fill=none] {$\alpha_{{\color{cyanind}\bullet}{\color{goldenyellow}\bullet}}$} (B3);
            \path (R21) edge  (B4);
            \path (R21) edge  (B5);
            
            \path (R22) edge  (B3);
            \path (R22) edge  (B4);
            \path (R22) edge[-,>=stealth',shorten >=1pt,auto,draw=gray,line width=1.5pt, fill=gray, opacity=1] node[anchor=center, fill=none] {$\alpha_{{\color{cyanind}\bullet}{\color{peach}\bullet}}$} (B5);
    
            \path (R31) edge[-,>=stealth',shorten >=1pt,auto,draw=gray,line width=1.5pt, fill=gray, opacity=1] node[anchor=center, fill=none] {$\alpha_{{\color{electricblue}\bullet}{\color{peach}\bullet}}$} (B5);
    
        \end{tikzpicture}
    \caption{Exemple de LBM}
    \label{fig:LBMvisu}
    \end{figure}

\end{frame}
\begin{frame}
    \frametitle{Extension de \emph{colSBM} aux réseaux bipartites}
    Le modèle \emph{colSBM} \parencite{chabert-liddellLearningCommonStructures2023}

    \begin{equation*}
        \overbrace{
        }^\text{Collection}
    \end{equation*}
    

\end{frame}
\begin{frame}
    \frametitle{Collections en bipartites}


\end{frame}

\section*{Bibliographie}
\begin{frame}
    \frametitle{Bibliographie}
    \printbibliography
\end{frame}

\end{document}