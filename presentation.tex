\documentclass{beamer}
\usetheme{Boadilla}

% importations
\usepackage[french]{babel} % pour dire que le texte est en francais
\usepackage{csquotes}
\usepackage[T1]{fontenc} % pour les font postscript
\usepackage[cyr]{aeguill} % Police vectorielle TrueType, guillemets francais
\usepackage{epsfig} % pour gérer les images
\usepackage{amsmath,amsthm} % très bon mode mathématique
\usepackage{amsfonts,amssymb,bm, bbold}% permet la definition des ensembles
\usepackage{algorithm2e} % pour les algorithmes
\usepackage{algpseudocode} % pour les algorithmes
\usepackage{float} % pour le placement des figure
\usepackage{url} % pour une gestion efficace des url
\usepackage{hyperref}  % pour les hyperliens dans le document
\usepackage{tikz} % For graph plots
% bibliographie
\usepackage[style=apa,sorting=none]{biblatex}
\addbibresource{references.bib}


% Tikz
%% Tikz Related
\usetikzlibrary{calc,shapes,backgrounds,arrows,automata,shadows,positioning}
\usetikzlibrary{arrows,shapes,positioning,shadows,trees,calc,backgrounds,automata,positioning}
\usetikzlibrary{decorations.pathreplacing,calligraphy}



\tikzset{
    basic/.style  = {draw, text width=3cm, font=\sffamily, rectangle},
    root/.style   = {basic, rounded corners=2pt, thin, align=center,
            fill=green!30},
    level 2/.style = {basic, rounded corners=6pt, thin,align=center, fill=green!60,
            text width=8em},
    level 3/.style = {basic, thin, align=left, fill=pink!60, text width=3.5cm}
}
% Couleurs
% pour tickz multilevel
\definecolor{redorg}{RGB}{215,48,39}
\definecolor{orangeorg}{RGB}{253,174,97}

\definecolor{blueind}{RGB}{69,117,233}
\definecolor{cyanind}{RGB}{116,173,209}
\definecolor{electricblue}{RGB}{125, 249, 255}

\definecolor{greenind}{RGB}{112,130,56}

\definecolor{burntorange}{RGB}{204, 85, 0}
\definecolor{goldenyellow}{RGB}{255, 192, 0}
\definecolor{peach}{RGB}{255, 229, 180}

\definecolor{gray}{RGB}{128,128,128}


\title{Séminaire des stagiaires}
\subtitle{Adaptation de colSBM aux réseaux bipartites}
\author[L. Lacoste]{Louis \textsc{Lacoste}}
\date{29 juin 2023}

\begin{document}

% titre
\begin{frame}[noframenumbering,plain]
    \maketitle
\end{frame}

\section{Contexte du modèle}
\begin{frame}
    \frametitle{Contexte écologique}
    \begin{itemize}
        \item De nombreux réseaux disponibles \parencite{WebLifeEcological} et décrivant des interactions similaires
        \item Re-grouper les réseaux selon leur similarité (\emph{clustering} de réseaux)
        \item Compléter d'éventuelles informations manquantes grâce à la collection
        \item Déterminer des structures d'interactions fines de manière agnostique
        \item Vérifier si le regroupement est lié à des co-variables
    \end{itemize}
    \footnotetext[0]{Pour combler les lacunes de\\\cite{chabert-liddellLearningCommonStructures2023}}
\end{frame}
\begin{frame}
    \frametitle{Réseaux bipartites\footnote{Ou \emph{bipartis}. Voir \cite{larousseDefinitionsBipartiBipartite}.}}
    \begin{columns}[c]
        \begin{column}{0.48\textwidth}
            \centering
            Réseau bipartite\\
            \begin{tikzpicture}[scale=.6]
                \tikzstyle{every edge}=[-,>=stealth',shorten >=1pt,auto,draw,line width=1.5pt]
                \tikzstyle{every state}=[draw, text=white,scale=0.95, transform shape]
                \tikzstyle{every state}=[draw=none,text=white,scale=0.75, transform shape]
                \tikzstyle{every node}=[fill=blueind]

                \node[state, draw=black!50] (A1) at (0,5) {\textbf{R1}};
                \node[state, draw=black!50] (A2) at (2.5,5) {\textbf{R2}};
                \node[state, draw=black!50] (A3) at (5,5) {\textbf{R3}};

                \tikzstyle{every node}=[fill=greenind, shape=rectangle]
                \tikzstyle{every state}=[draw=none,text=white,scale=0.75, transform shape, shape=rectangle]
                \node[state, draw=black!50] (B1) at (0,0) {\textbf{C1}};
                \node[state, draw=black!50] (B2) at (1.25,0) {\textbf{C2}};
                \node[state, draw=black!50] (B3) at (2.5,0) {\textbf{C3}};
                \node[state, draw=black!50] (B4) at (3.75,0) {\textbf{C4}};
                \node[state, draw=black!50] (B5) at (5,0) {\textbf{C5}};
                \path (A1) edge [] (B1);
                \path (A1) edge  (B2);
                \path (A1) edge  (B3);
                \path (A1) edge  (B4);
                \path (A2) edge  (B3);
                \path (A2) edge  (B4);
                \path (A3) edge  (B5);
                \path (A2) edge  (B5);
            \end{tikzpicture}
        \end{column}
        \hfill
        \begin{column}{0.48\linewidth}
            Matrice d'incidence
            \smallskip
            $B=\left(
                \begin{array}{rrrrr}
                        1 & 1 & 1 & 1 & 0 \\
                        0 & 0 & 1 & 1 & 1 \\
                        0 & 0 & 0 & 0 & 1 \\
                    \end{array}\right)
            $\\
        \end{column}
    \end{columns}
    \smallskip
    Permet de décrire des interactions impliquant deux agents dont les rôles
    sont de natures différentes.\\
    Par exemple : hôtes-parasites, plantes-pollinisateurs, graines-disperseurs \dots
\end{frame}
\begin{frame}
    \frametitle{Latent Block Model (LBM)}
    Proposé par \cite{govaertEMAlgorithmBlock2005}.
    \begin{columns}
        \begin{column}{0.5\linewidth}
            \begin{figure}[H]
                \center
                \begin{tikzpicture}[scale=.45]
                    \tikzstyle{every state}=[draw, text=white,scale=0.95, transform shape]
                    \tikzstyle{every state}=[draw=none,text=white,scale=0.75, transform shape]
                    \tikzset{edge_proba/.style={draw=white, fill=none, text=black}}

                    \tikzstyle{every node}=[fill=blueind]
                    \node[edge_proba] (pi1) at (1,5.7) {\textbf{$\pi_{{\color{blueind}\bullet}}$}};
                    \node[state, draw=black!50] (R11) at (0,5) {\textbf{R11}};
                    \node[state, draw=black!50] (R12) at (1,5) {\textbf{R12}};
                    \node[state, draw=black!50] (R13) at (2,5) {\textbf{R13}};

                    \tikzstyle{every node}=[fill=cyanind]
                    \node[edge_proba] (pi2) at (6.75,5.7) {\textbf{$\pi_{{\color{cyanind}\bullet}}$}};
                    \node[state, draw=black!50] (R21) at (6.25,5) {\textbf{R21}};
                    \node[state, draw=black!50] (R22) at (7.25,5) {\textbf{R22}};

                    \tikzstyle{every node}=[fill=electricblue]
                    \node[edge_proba] (pi3) at (10,5.7) {\textbf{$\pi_{{\color{electricblue}\bullet}}$}};
                    \node[state, draw=black!50] (R31) at (10,5) {\textbf{R31}};

                    \tikzstyle{every node}=[fill=burntorange, shape=rectangle]
                    \node[edge_proba] (pi3) at (0.5,-0.7) {\textbf{$\rho_{{\color{burntorange}\bullet}}$}};
                    \tikzstyle{every state}=[draw=none,text=white,scale=0.75, transform shape, shape=rectangle]
                    \node[state, draw=black!50] (B1) at (0,0) {\textbf{C11}};
                    \node[state, draw=black!50] (B2) at (1,0) {\textbf{C12}};
                    \tikzstyle{every node}=[fill=goldenyellow, shape=rectangle]
                    \node[edge_proba] (pi3) at (4,-0.7) {\textbf{$\rho_{{\color{goldenyellow}\bullet}}$}};
                    \node[state, draw=black!50] (B3) at (3.5,0) {\textbf{C21}};
                    \node[state, draw=black!50] (B4) at (4.5,0) {\textbf{C22}};
                    \tikzstyle{every node}=[fill=peach, shape=rectangle]
                    \node[edge_proba] (pi3) at (10,-0.7) {\textbf{$\rho_{{\color{peach}\bullet}}$}};
                    \node[state, draw=black!50] (B5) at (10,0) {\textbf{C31}};

                    \tikzstyle{every edge}=[-,>=stealth',shorten >=1pt,auto,draw,line width=1.5pt,draw opacity=0.2]

                    \path (R11) edge[-,>=stealth',shorten >=1pt,auto,draw=gray,line width=1.5pt, fill=gray, opacity=1] node[left, fill=none] {$\alpha_{{\color{blueind}\bullet}{\color{burntorange}\bullet}}$} (B1);
                    \path (R11) edge (B2);
                    \path (R11) edge  (B3);
                    \path (R11) edge  (B4);

                    \path (R12) edge [] (B1);
                    \path (R12) edge  (B2);
                    \path (R12) edge  (B3);
                    \path (R12) edge  (B4);

                    \path (R13) edge [] (B1);
                    \path (R13) edge  (B2);
                    \path (R13) edge  (B3);
                    \path (R13) edge[-,>=stealth',shorten >=1pt,auto,draw=gray,line width=1.5pt, fill=gray, opacity=1] node[midway, left, fill=none] {$\alpha_{{\color{blueind}\bullet}{\color{goldenyellow}\bullet}}$} (B4);

                    \path (R21) edge[-,>=stealth',shorten >=1pt,auto,draw=gray,line width=1.5pt, fill=gray, opacity=1] node[midway, right, fill=none] {$\alpha_{{\color{cyanind}\bullet}{\color{goldenyellow}\bullet}}$} (B3);
                    \path (R21) edge  (B4);
                    \path (R21) edge  (B5);

                    \path (R22) edge  (B3);
                    \path (R22) edge  (B4);
                    \path (R22) edge[-,>=stealth',shorten >=1pt,auto,draw=gray,line width=1.5pt, fill=gray, opacity=1] node[midway, left, fill=none] {$\alpha_{{\color{cyanind}\bullet}{\color{peach}\bullet}}$} (B5);

                    \path (R31) edge[-,>=stealth',shorten >=1pt,auto,draw=gray,line width=1.5pt, fill=gray, opacity=1] node[midway, right, fill=none] {$\alpha_{{\color{electricblue}\bullet}{\color{peach}\bullet}}$} (B5);

                \end{tikzpicture}
                \caption{Exemple de LBM}
                \label{fig:LBMvisu}
            \end{figure}
        \end{column}
        \begin{column}{0.5\linewidth}
            \begin{block}{Paramètres}
                \begin{itemize}
                    \item $\mathcal{K}_1 = \{{\color{blueind}\bullet},{\color{cyanind}\bullet},{\color{electricblue}\bullet}\}$ blocs en ligne
                    \item $\mathcal{K}_2 = \{{\color{burntorange}\bullet},{\color{goldenyellow}\bullet},{\color{peach}\bullet}\}$ blocs en colonne
                    \item $\pi_{\bullet} = \mathbb{P}(i\in\bullet)$ en ligne et $\rho_{\bullet} = \mathbb{P}(j\in\bullet)$ en colonne
                    \item $\alpha_{{\color{blueind}\bullet}{\color{burntorange}\bullet}} = \mathbb{P}(i \leftrightarrow j | i \in {\color{blueind}\bullet}, j \in {\color{burntorange}\bullet})$
                \end{itemize}
            \end{block}
        \end{column}
    \end{columns}


\end{frame}
\begin{frame}
    \frametitle{\emph{colSBM}}
    Le modèle \emph{colSBM} \parencite{chabert-liddellLearningCommonStructures2023}.\\
    \smallskip

    \definecolor{yellow}{RGB}{255,190,60}
    \begin{tikzpicture}[scale=.33]
        \tikzstyle{every edge}=[-,>=stealth',shorten >=1pt,auto,draw,line width=.5pt, bend left]
        \tikzstyle{every state}=[draw, text=white,scale=0.95, transform shape]
        \tikzset{edge_proba/.style={draw=white, fill=none, text=black}}

        \tikzstyle{every node}=[fill=yellow]
        \node[state, draw=black!50] (A1) at (0,2) {\textbf{A1}};
        \node[state, draw=black!50] (A2) at (1.5, 2) {\textbf{A2}};
        \node[state, draw=black!50] (A3) at (0.75,3.25) {\textbf{A3}};

        \tikzstyle{every node}=[fill=blueind]
        \node[state, draw=black!50] (B1) at (4.5,3) {\textbf{B1}};
        \node[state, draw=black!50] (B2) at (4,4.75) {\textbf{B2}};
        \node[state, draw=black!50] (B3) at (5.5,6) {\textbf{B3}};
        \node[state, draw=black!50] (B4) at (7,4.75) {\textbf{B4}};
        \node[state, draw=black!50] (B5) at (6.5,3) {\textbf{B5}};

        \tikzstyle{every node}=[fill=greenind]
        \node[state, draw=black!50] (C1) at (5,0) {\textbf{C1}};
        \node[state, draw=black!50] (C2) at (7,1) {\textbf{C2}};


        \path (A1) edge[bend right] (A2);
        \path (A1) edge node[midway, left, fill=none] {$\alpha_{{\color{yellow}\bullet}{\color{yellow}\bullet}}$} (A3);
        \path (A3) edge (A2);

        \path (A3) edge node[midway, above, fill=none] {$\alpha_{{\color{yellow}\bullet}{\color{blueind}\bullet}}$} (B3);

        \path (B1) edge (B2);
        \path (B2) edge (B3);
        \path (B3) edge (B4);
        \path (B4) edge (B5);
        \path (B5) edge (B1);

        \path (B1) edge[bend left=0] (B4);
        \path (B5) edge[bend left=0] (B2);

        \path (A2) edge[bend right] node[midway, below, fill=none] {$\alpha_{{\color{yellow}\bullet}{\color{greenind}\bullet}}$} (C1);
        \path (C1) edge[bend right] node[midway, below, fill=none] {$\alpha_{{\color{greenind}\bullet}{\color{greenind}\bullet}}$} (C2);
        \path (C2) edge[bend right] node[midway, right, fill=none] {$\alpha_{{\color{greenind}\bullet}{\color{blueind}\bullet}}$} (B4);

        \node[font=\small, text justified,draw=none, fill=none] at (4.5,-1.5) {SBM};

        \node[font=\small, text justified, fill=none] at (11.5, 1.5) {$\Longrightarrow$};

        % Sampled network
        \begin{scope}[xshift=14.5cm, yshift=4cm]
            \tikzstyle{every node}=[fill=gray, scale=0.95]
            \tikzstyle{every edge}=[-,>=stealth',shorten >=1pt,auto,draw,line width=.5pt, bend left]
            \tikzstyle{every state}=[draw, text=white,scale=0.95, transform shape]

            \node[state, draw=black!50] (A1) at (0,0) {\textbf{10}};
            \node[state, draw=black!50] (A2) at (1, 0) {\textbf{2}};
            \node[state, draw=black!50] (A3) at (0.5,1) {\textbf{5}};

            \node[state, draw=black!50] (B1) at (2.5,1) {\textbf{1}};
            \node[state, draw=black!50] (B2) at (2,2.75) {\textbf{9}};
            \node[state, draw=black!50] (B3) at (3.5,4) {\textbf{6}};
            \node[state, draw=black!50] (B4) at (5,2.75) {\textbf{3}};
            \node[state, draw=black!50] (B5) at (4.5,1) {\textbf{7}};

            \node[state, draw=black!50] (C1) at (3,-0.5) {\textbf{4}};
            \node[state, draw=black!50] (C2) at (5,0) {\textbf{8}};

            \path (A1) edge[bend right] (A2);
            \path (A1) edge (A3);
            \path (A3) edge (A2);

            \path (A3) edge (B3);

            \path (B1) edge (B2);
            \path (B2) edge (B3);
            \path (B3) edge (B4);
            \path (B4) edge (B5);
            \path (B5) edge (B1);

            \path (B1) edge[bend left=0] (B4);
            \path (B5) edge[bend left=0] (B2);

            \path (A2) edge[bend right] (C1);
            \path (C1) edge[bend right] (C2);
            \path (C2) edge[bend right] (B4);

            \node[text width=3cm,font=\small, text justified, rotate=90, fill=none, below = -0.8cm of C1] (dots) {\dots};
            \draw [decorate, decoration = {brace}] (6, 4) --  (6,-8.5);
            \node[text width=3cm, font=\small, text justified, fill=none] at (11.5,-2.25) {$M$ réalisations indépendantes du SBM};


        \end{scope}
        \begin{scope}[xshift=14.5cm, yshift=-4cm]
            \tikzstyle{every node}=[fill=gray, scale=0.95]
            \tikzstyle{every edge}=[-,>=stealth',shorten >=1pt,auto,draw,line width=.5pt, bend left]
            \tikzstyle{every state}=[draw, text=white,scale=0.95, transform shape]

            \node[state, draw=black!50] (A1) at (0,0) {\textbf{9}};
            \node[state, draw=black!50] (A2) at (1, 0) {\textbf{2}};
            \node[state, draw=black!50] (A3) at (0.5,1) {\textbf{1}};

            \node[state, draw=black!50] (B1) at (2.5,1) {\textbf{5}};
            \node[state, draw=black!50] (B2) at (2,2.75) {\textbf{10}};
            \node[state, draw=black!50] (B3) at (3.5,4) {\textbf{4}};
            \node[state, draw=black!50] (B4) at (5,2.75) {\textbf{8}};
            \node[state, draw=black!50] (B5) at (4.5,1) {\textbf{7}};

            \node[state, draw=black!50] (C1) at (3,-0.5) {\textbf{6}};
            \node[state, draw=black!50] (C2) at (5,0) {\textbf{3}};


            \path (A1) edge[bend right] (A2);
            \path (A1) edge (A3);
            \path (A3) edge (A2);

            \path (A3) edge (B3);

            \path (B1) edge (B2);
            \path (B2) edge (B3);
            \path (B3) edge (B4);
            \path (B4) edge (B5);
            \path (B5) edge (B1);

            \path (B1) edge[bend left=0] (B4);
            \path (B5) edge[bend left=0] (B2);

            \path (A2) edge[bend right] (C1);
            \path (C1) edge[bend right] (C2);
            \path (C2) edge[bend right] (B4);
        \end{scope}
    \end{tikzpicture}
    \begin{columns}
        \begin{column}{0.48\linewidth}
            \begin{block}{Paramètres}
                \begin{itemize}
                    \item $\mathcal{K} = \{{\color{yellow}\bullet},{\color{blueind}\bullet},{\color{greenind}\bullet}\}$ blocs
                    \item $\pi_{\bullet} = \mathbb{P}(i\in\bullet)$
                    \item $\alpha_{{\color{greenind}\bullet}{\color{blueind}\bullet}} = \mathbb{P}(i \leftrightarrow j | i \in {\color{greenind}\bullet}, j \in {\color{blueind}\bullet})$
                \end{itemize}
            \end{block}
        \end{column}
        \begin{column}{0.52\linewidth}
        \end{column}
    \end{columns}

\end{frame}
\section{Extension de \emph{colSBM} aux réseaux bipartites}
\begin{frame}
    \frametitle{Collections bipartites}
    \begin{columns}
        \begin{column}{0.5\linewidth}
            \begin{tikzpicture}[scale=.38]
                \tikzstyle{every state}=[draw, text=white,scale=0.95, transform shape]
                \tikzstyle{every state}=[draw=none,text=white,scale=0.75, transform shape]
                \tikzset{edge_proba/.style={draw=white, fill=none, text=black}}

                \tikzstyle{every node}=[fill=blueind]
                \node[edge_proba] (pi1) at (1,5.7) {\textbf{$\pi_{{\color{blueind}\bullet}}$}};
                \node[state, draw=black!50] (R11) at (0,5) {\textbf{R11}};
                \node[state, draw=black!50] (R12) at (1,5) {\textbf{R12}};
                \node[state, draw=black!50] (R13) at (2,5) {\textbf{R13}};

                \tikzstyle{every node}=[fill=cyanind]
                \node[edge_proba] (pi2) at (6.75,5.7) {\textbf{$\pi_{{\color{cyanind}\bullet}}$}};
                \node[state, draw=black!50] (R21) at (6.25,5) {\textbf{R21}};
                \node[state, draw=black!50] (R22) at (7.25,5) {\textbf{R22}};

                \tikzstyle{every node}=[fill=electricblue]
                \node[edge_proba] (pi3) at (10,5.7) {\textbf{$\pi_{{\color{electricblue}\bullet}}$}};
                \node[state, draw=black!50] (R31) at (10,5) {\textbf{R31}};

                \tikzstyle{every node}=[fill=burntorange, shape=rectangle]
                \node[edge_proba] (pi3) at (0.5,-0.7) {\textbf{$\rho_{{\color{burntorange}\bullet}}$}};
                \tikzstyle{every state}=[draw=none,text=white,scale=0.75, transform shape, shape=rectangle]
                \node[state, draw=black!50] (B1) at (0,0) {\textbf{C11}};
                \node[state, draw=black!50] (B2) at (1,0) {\textbf{C12}};
                \tikzstyle{every node}=[fill=goldenyellow, shape=rectangle]
                \node[edge_proba] (pi3) at (4,-0.7) {\textbf{$\rho_{{\color{goldenyellow}\bullet}}$}};
                \node[state, draw=black!50] (B3) at (3.5,0) {\textbf{C21}};
                \node[state, draw=black!50] (B4) at (4.5,0) {\textbf{C22}};
                \tikzstyle{every node}=[fill=peach, shape=rectangle]
                \node[edge_proba] (pi3) at (10,-0.7) {\textbf{$\rho_{{\color{peach}\bullet}}$}};
                \node[state, draw=black!50] (B5) at (10,0) {\textbf{C31}};

                \tikzstyle{every edge}=[-,>=stealth',shorten >=1pt,auto,draw,line width=1.5pt,draw opacity=0.2]

                \path (R11) edge[-,>=stealth',shorten >=1pt,auto,draw=gray,line width=1.5pt, fill=gray, opacity=1] node[left, fill=none] {$\alpha_{{\color{blueind}\bullet}{\color{burntorange}\bullet}}$} (B1);
                \path (R11) edge (B2);
                \path (R11) edge  (B3);
                \path (R11) edge  (B4);

                \path (R12) edge [] (B1);
                \path (R12) edge  (B2);
                \path (R12) edge  (B3);
                \path (R12) edge  (B4);

                \path (R13) edge [] (B1);
                \path (R13) edge  (B2);
                \path (R13) edge  (B3);
                \path (R13) edge[-,>=stealth',shorten >=1pt,auto,draw=gray,line width=1.5pt, fill=gray, opacity=1] node[midway, left, fill=none] {$\alpha_{{\color{blueind}\bullet}{\color{goldenyellow}\bullet}}$} (B4);

                \path (R21) edge[-,>=stealth',shorten >=1pt,auto,draw=gray,line width=1.5pt, fill=gray, opacity=1] node[midway, right, fill=none] {$\alpha_{{\color{cyanind}\bullet}{\color{goldenyellow}\bullet}}$} (B3);
                \path (R21) edge  (B4);
                \path (R21) edge  (B5);

                \path (R22) edge  (B3);
                \path (R22) edge  (B4);
                \path (R22) edge[-,>=stealth',shorten >=1pt,auto,draw=gray,line width=1.5pt, fill=gray, opacity=1] node[midway, left, fill=none] {$\alpha_{{\color{cyanind}\bullet}{\color{peach}\bullet}}$} (B5);

                \path (R31) edge[-,>=stealth',shorten >=1pt,auto,draw=gray,line width=1.5pt, fill=gray, opacity=1] node[midway, right, fill=none] {$\alpha_{{\color{electricblue}\bullet}{\color{peach}\bullet}}$} (B5);

            \end{tikzpicture}
            \begin{block}{Paramètres}
                \begin{itemize}
                    \item $\mathcal{K}_1 = \{{\color{blueind}\bullet},{\color{cyanind}\bullet},{\color{electricblue}\bullet}\}$ blocs en ligne
                    \item $\mathcal{K}_2 = \{{\color{burntorange}\bullet},{\color{goldenyellow}\bullet},{\color{peach}\bullet}\}$ blocs en colonne
                    \item $\pi_{\bullet} = \mathbb{P}(i\in\bullet)$ en ligne et $\rho_{\bullet} = \mathbb{P}(j\in\bullet)$ en colonne
                    \item $\alpha_{{\color{blueind}\bullet}{\color{burntorange}\bullet}} = \mathbb{P}(i \leftrightarrow j | i \in {\color{blueind}\bullet}, j \in {\color{burntorange}\bullet})$
                \end{itemize}
            \end{block}
        \end{column}
        \begin{column}{0.5\linewidth}
            \centering
            \begin{tikzpicture}[scale=0.6]
                \begin{scope}[yshift = 4cm]
                    \tikzstyle{every state}=[draw, text=white,scale=0.75, transform shape]

                    \tikzstyle{every node}=[fill=gray]
                    \node[state, draw=black!50] (R11) at (0,1.25) {\textbf{1}};
                    \node[state, draw=black!50] (R12) at (1,1.25) {\textbf{2}};
                    \node[state, draw=black!50] (R13) at (2,1.25) {\textbf{3}};
                    \node[state, draw=black!50] (R21) at (3,1.25) {\textbf{4}};
                    \node[state, draw=black!50] (R22) at (4,1.25) {\textbf{5}};
                    \node[state, draw=black!50] (R31) at (5,1.25) {\textbf{6}};

                    \tikzstyle{every state}=[draw=none,text=white,scale=0.75, transform shape, shape=rectangle]
                    \node[state, draw=black!50] (B1) at (0.5,0) {\textbf{1}};
                    \node[state, draw=black!50] (B2) at (1.5,0) {\textbf{2}};

                    \node[state, draw=black!50] (B31) at (2.5,0) {\textbf{3}};
                    \node[state, draw=black!50] (B4) at (3.5,0) {\textbf{4}};

                    \node[state, draw=black!50] (B5) at (4.5,0) {\textbf{5}};

                    \tikzstyle{every edge}=[-,>=stealth',shorten >=1pt,auto,draw,line width=1pt, draw=gray, fill=gray]
                    \path (R11) edge (B1);
                    \path (R11) edge (B2);
                    \path (R11) edge  (B31);
                    \path (R11) edge  (B4);

                    \path (R12) edge [] (B1);
                    \path (R12) edge  (B2);
                    \path (R12) edge  (B31);
                    \path (R12) edge  (B4);

                    \path (R13) edge [] (B1);
                    \path (R13) edge  (B2);
                    \path (R13) edge  (B31);
                    \path (R13) edge (B4);

                    \path (R21) edge (B31);
                    \path (R21) edge  (B4);
                    \path (R21) edge  (B5);

                    \path (R22) edge  (B31);
                    \path (R22) edge  (B4);
                    \path (R22) edge (B5);

                    \path (R31) edge (B5);
                \end{scope}
                \node[text width=3cm,font=\small, text justified, rotate=90, fill=none] (dots) at (2.5, 4.75){\dots};

                \begin{scope}[yshift = 0cm]
                    \tikzstyle{every state}=[draw, text=white,scale=0.75, transform shape]

                    \tikzstyle{every node}=[fill=gray]
                    \node[state, draw=black!50] (R11) at (0,1.25) {\textbf{4}};
                    \node[state, draw=black!50] (R12) at (1,1.25) {\textbf{1}};
                    \node[state, draw=black!50] (R13) at (2,1.25) {\textbf{6}};
                    \node[state, draw=black!50] (R21) at (3,1.25) {\textbf{3}};
                    \node[state, draw=black!50] (R22) at (4,1.25) {\textbf{5}};
                    \node[state, draw=black!50] (R31) at (5,1.25) {\textbf{2}};

                    \tikzstyle{every state}=[draw=none,text=white,scale=0.75, transform shape, shape=rectangle]
                    \node[state, draw=black!50] (B1) at (0.5,0) {\textbf{5}};
                    \node[state, draw=black!50] (B2) at (1.5,0) {\textbf{1}};

                    \node[state, draw=black!50] (B3) at (2.5,0) {\textbf{3}};
                    \node[state, draw=black!50] (B4) at (3.5,0) {\textbf{2}};

                    \node[state, draw=black!50] (B5) at (4.5,0) {\textbf{4}};

                    \tikzstyle{every edge}=[-,>=stealth',shorten >=1pt,auto,draw,line width=1pt, draw=gray, fill=gray]
                    \path (R11) edge (B1);
                    \path (R11) edge (B2);
                    \path (R11) edge  (B3);
                    \path (R11) edge  (B4);

                    \path (R12) edge [] (B1);
                    \path (R12) edge  (B2);
                    \path (R12) edge  (B3);
                    \path (R12) edge  (B4);

                    \path (R13) edge [] (B1);
                    \path (R13) edge  (B2);
                    \path (R13) edge  (B3);
                    \path (R13) edge (B4);

                    \path (R21) edge (B3);
                    \path (R21) edge  (B4);
                    \path (R21) edge  (B5);

                    \path (R22) edge  (B3);
                    \path (R22) edge  (B4);
                    \path (R22) edge (B5);

                    \path (R31) edge (B5);
                \end{scope}

            \end{tikzpicture}
        \end{column}
    \end{columns}

\end{frame}

\section*{Bibliographie}
\begin{frame}[noframenumbering,plain,allowframebreaks]
    \frametitle{Bibliographie}
    \hfill
    \begin{minipage}{0.9\textwidth}
        \printbibliography
    \end{minipage}
\end{frame}

\end{document}