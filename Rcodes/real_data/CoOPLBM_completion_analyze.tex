\hypertarget{context-of-this-analysis}{%
\subsubsection{Context of this
analysis}\label{context-of-this-analysis}}

After performing a netclustering on the raw data, we will see if the
detect structure resulting in the clustering comes from the sampling
effort. To test this we will use the CoOPLBM model by
\cite{anakokDisentanglingStructureEcological2022} to complete the data.

The CoOPLBM model assumes that the observed incidence matrix \(R\) is an
element-wise product of an \(M\) matrix following an LBM and an \(N\)
matrix which elements follow Poisson distributions independent on \(M\).

The model gives us the \(\widehat{M}\) matrix, the elements of which
are:

\[\widehat{M_{ij}} = \mathbb{P}(M_{ij} = 1)\]

Note that if \(R_{ij} = 1\) then \(\widehat{M_{ij}} = 1\)

\begin{itemize}
\tightlist
\item
  1 if the interaction was observed
\item
  a probability, that there should be an interaction but it wasn't
  observed
\end{itemize}

This \emph{completed matrix} can be used in different manners to be fed
to the colSBM model.

\hypertarget{threshold-based-completions}{%
\subsubsection{Threshold based
completions}\label{threshold-based-completions}}

With the thresholds, the infered incidence matrix obtained by CoOPLBM is
used to generate a completed incidence matrix by the following procedure
: \[X_{ij} = \begin{cases}
  1 & \text{if the value is over the threshold} \\
  0 & \text{else} \\
\end{cases}\]

\hypertarget{completed-threshold}{%
\paragraph{0.5 completed threshold}\label{completed-threshold}}

Here, the completion threshold is set to \(0.5\).

First we will compute an ARI on the collection id given by the raw data
and the completed matrix.

\begin{longtable}[]{@{}lr@{}}
\toprule
& ARI with uncompleted data\tabularnewline
\midrule
\endhead
iid & 0.1142823\tabularnewline
pi & 0.0263660\tabularnewline
rho & 0.0933340\tabularnewline
pirho & 0.2158747\tabularnewline
\bottomrule
\end{longtable}

In the above table, one can see the network clustering obtained after
applying CoOPLBM has not much in common with the clustering of the
uncompleted data.

\hypertarget{number-of-sub-collections-and-details-of-each-sub-collection}{%
\subparagraph{Number of sub-collections and details of each
sub-collection}\label{number-of-sub-collections-and-details-of-each-sub-collection}}

\hypertarget{completed-threshold-1}{%
\subsubsection{0.2 completed threshold}\label{completed-threshold-1}}

The \(0.2\) threshold adds a lot of interactions compared to raw matrix.

\begin{longtable}[]{@{}lr@{}}
\toprule
& ARI with uncompleted data\tabularnewline
\midrule
\endhead
iid & 0.0429465\tabularnewline
pi & 0.0330057\tabularnewline
rho & 0.0187305\tabularnewline
pirho & 0.0357728\tabularnewline
\bottomrule
\end{longtable}

Same as for \(0.5\), after applying CoOPLBM the obtained clustering
doesn't match the uncompleted data.

\hypertarget{sample-based-completions}{%
\subsubsection{Sample based
completions}\label{sample-based-completions}}

The \(M\) matrix is used to sample a new \(X\) matrix which elements are
the realisation of Bernoulli distributions of probability \(M_{i,j}\).
\[\mathbb{P}(X_{i,j} = 1) = M_{i,j} \]

\begin{longtable}[]{@{}lr@{}}
\toprule
& ARI with uncompleted data\tabularnewline
\midrule
\endhead
iid & 0.0148172\tabularnewline
pi & 0.0265793\tabularnewline
rho & 0.0051536\tabularnewline
pirho & 0.0152299\tabularnewline
\bottomrule
\end{longtable}
